\documentclass[preprint, 3p,
authoryear]{elsarticle} %review=doublespace preprint=single 5p=2 column
%%% Begin My package additions %%%%%%%%%%%%%%%%%%%

\usepackage[hyphens]{url}

  \journal{GitHub} % Sets Journal name

\usepackage{lineno} % add

\usepackage{graphicx}
%%%%%%%%%%%%%%%% end my additions to header

\usepackage[T1]{fontenc}
\usepackage{lmodern}
\usepackage{amssymb,amsmath}
\usepackage{ifxetex,ifluatex}
\usepackage{fixltx2e} % provides \textsubscript
% use upquote if available, for straight quotes in verbatim environments
\IfFileExists{upquote.sty}{\usepackage{upquote}}{}
\ifnum 0\ifxetex 1\fi\ifluatex 1\fi=0 % if pdftex
  \usepackage[utf8]{inputenc}
\else % if luatex or xelatex
  \usepackage{fontspec}
  \ifxetex
    \usepackage{xltxtra,xunicode}
  \fi
  \defaultfontfeatures{Mapping=tex-text,Scale=MatchLowercase}
  \newcommand{\euro}{€}
\fi
% use microtype if available
\IfFileExists{microtype.sty}{\usepackage{microtype}}{}
\usepackage[]{natbib}
\bibliographystyle{plainnat}

\usepackage{graphicx}
\ifxetex
  \usepackage[setpagesize=false, % page size defined by xetex
              unicode=false, % unicode breaks when used with xetex
              xetex]{hyperref}
\else
  \usepackage[unicode=true]{hyperref}
\fi
\hypersetup{breaklinks=true,
            bookmarks=true,
            pdfauthor={},
            pdftitle={Tecnicas y Herramientas Modernas I - Curso de Programación en R},
            colorlinks=false,
            urlcolor=blue,
            linkcolor=magenta,
            pdfborder={0 0 0}}
\usepackage[normalem]{ulem}
% avoid problems with \sout in headers with hyperref:
\pdfstringdefDisableCommands{\renewcommand{\sout}{}}

\setcounter{secnumdepth}{5}
% Pandoc toggle for numbering sections (defaults to be off)

% Pandoc syntax highlighting
\usepackage{color}
\usepackage{fancyvrb}
\newcommand{\VerbBar}{|}
\newcommand{\VERB}{\Verb[commandchars=\\\{\}]}
\DefineVerbatimEnvironment{Highlighting}{Verbatim}{commandchars=\\\{\}}
% Add ',fontsize=\small' for more characters per line
\usepackage{framed}
\definecolor{shadecolor}{RGB}{248,248,248}
\newenvironment{Shaded}{\begin{snugshade}}{\end{snugshade}}
\newcommand{\AlertTok}[1]{\textcolor[rgb]{0.94,0.16,0.16}{#1}}
\newcommand{\AnnotationTok}[1]{\textcolor[rgb]{0.56,0.35,0.01}{\textbf{\textit{#1}}}}
\newcommand{\AttributeTok}[1]{\textcolor[rgb]{0.77,0.63,0.00}{#1}}
\newcommand{\BaseNTok}[1]{\textcolor[rgb]{0.00,0.00,0.81}{#1}}
\newcommand{\BuiltInTok}[1]{#1}
\newcommand{\CharTok}[1]{\textcolor[rgb]{0.31,0.60,0.02}{#1}}
\newcommand{\CommentTok}[1]{\textcolor[rgb]{0.56,0.35,0.01}{\textit{#1}}}
\newcommand{\CommentVarTok}[1]{\textcolor[rgb]{0.56,0.35,0.01}{\textbf{\textit{#1}}}}
\newcommand{\ConstantTok}[1]{\textcolor[rgb]{0.00,0.00,0.00}{#1}}
\newcommand{\ControlFlowTok}[1]{\textcolor[rgb]{0.13,0.29,0.53}{\textbf{#1}}}
\newcommand{\DataTypeTok}[1]{\textcolor[rgb]{0.13,0.29,0.53}{#1}}
\newcommand{\DecValTok}[1]{\textcolor[rgb]{0.00,0.00,0.81}{#1}}
\newcommand{\DocumentationTok}[1]{\textcolor[rgb]{0.56,0.35,0.01}{\textbf{\textit{#1}}}}
\newcommand{\ErrorTok}[1]{\textcolor[rgb]{0.64,0.00,0.00}{\textbf{#1}}}
\newcommand{\ExtensionTok}[1]{#1}
\newcommand{\FloatTok}[1]{\textcolor[rgb]{0.00,0.00,0.81}{#1}}
\newcommand{\FunctionTok}[1]{\textcolor[rgb]{0.00,0.00,0.00}{#1}}
\newcommand{\ImportTok}[1]{#1}
\newcommand{\InformationTok}[1]{\textcolor[rgb]{0.56,0.35,0.01}{\textbf{\textit{#1}}}}
\newcommand{\KeywordTok}[1]{\textcolor[rgb]{0.13,0.29,0.53}{\textbf{#1}}}
\newcommand{\NormalTok}[1]{#1}
\newcommand{\OperatorTok}[1]{\textcolor[rgb]{0.81,0.36,0.00}{\textbf{#1}}}
\newcommand{\OtherTok}[1]{\textcolor[rgb]{0.56,0.35,0.01}{#1}}
\newcommand{\PreprocessorTok}[1]{\textcolor[rgb]{0.56,0.35,0.01}{\textit{#1}}}
\newcommand{\RegionMarkerTok}[1]{#1}
\newcommand{\SpecialCharTok}[1]{\textcolor[rgb]{0.00,0.00,0.00}{#1}}
\newcommand{\SpecialStringTok}[1]{\textcolor[rgb]{0.31,0.60,0.02}{#1}}
\newcommand{\StringTok}[1]{\textcolor[rgb]{0.31,0.60,0.02}{#1}}
\newcommand{\VariableTok}[1]{\textcolor[rgb]{0.00,0.00,0.00}{#1}}
\newcommand{\VerbatimStringTok}[1]{\textcolor[rgb]{0.31,0.60,0.02}{#1}}
\newcommand{\WarningTok}[1]{\textcolor[rgb]{0.56,0.35,0.01}{\textbf{\textit{#1}}}}

% tightlist command for lists without linebreak
\providecommand{\tightlist}{%
  \setlength{\itemsep}{0pt}\setlength{\parskip}{0pt}}






\begin{document}


\begin{frontmatter}

  \title{Tecnicas y Herramientas Modernas I - Curso de Programación en
R}
    \author[Universidad Nacional de Cuyo]{Nicolas Roman Farina Lorenzo%
  \corref{cor1}%
  \fnref{1}}
   \ead{nicofarina1323@gmail.com} 
    \author[Universidad Nacional de Cuyo]{Juan Manuel Aspee}
   \ead{juanaspeerobert14@gmail.com} 
    \author[Universidad Nacional de Cuyo]{Joaquien Costa%
  %
  \fnref{2}}
   \ead{joacosta010@gmail.com} 
    \author[Universidad Nacional de Cuyo]{Lucas Romero%
  %
  \fnref{2}}
   \ead{lucasromero\_5@hotmail.com} 
      \affiliation[Universidad Nacional de Cuyo]{Centro Universitario,
M5500 Mendoza}
    \cortext[cor1]{Corresponding author}
    \fntext[1]{Este es el informe final del modulo R.}
    \fntext[2]{Esperemos les guste :).}
  
  \begin{abstract}
  En este Informe se describirán diferentes herramientas para medir el
  rendimiento de un código, además de varios metodos numéricos e
  implementaciones interesantes de código para simulacion de situaciones
  de la vida real.
  \end{abstract}
    \begin{keyword}
    rendimiento \sep performance \sep 
    codigo
  \end{keyword}
  
 \end{frontmatter}

\hypertarget{muxe9todos-para-medir-rendimiento-de-cuxf3digo-en-r}{%
\section{\texorpdfstring{\uline{\textbf{MÉTODOS PARA MEDIR RENDIMIENTO
DE CÓDIGO EN
R:}}}{MÉTODOS PARA MEDIR RENDIMIENTO DE CÓDIGO EN R:}}\label{muxe9todos-para-medir-rendimiento-de-cuxf3digo-en-r}}

A continuación se detallan algunas bibliotecas que utilizaremos a lo
largo del informe para medir el rendimiento de nuestros códigos:

\hypertarget{muxe9todo-sys.time}{%
\subsection{\texorpdfstring{\uline{\textbf{MÉTODO
SYS.TIME}}}{MÉTODO SYS.TIME}}\label{muxe9todo-sys.time}}

Este es un método muy simple y flexible de usar que viene preinstalado
en R, aun así, tiene algunas desventajas en su implementación
(tardaremos mucho en compilar un PDF o una presentación que contenga
dicho método). Nos dejara medir el rendimiento de código a través de los
registros del RTC (Real Time Clock) de la computadora. Un ejemplo podría
ser el siguiente:

\begin{Shaded}
\begin{Highlighting}[]
\NormalTok{duermete\_un\_minuto }\OtherTok{\textless{}{-}} \ControlFlowTok{function}\NormalTok{() \{ }\FunctionTok{Sys.sleep}\NormalTok{(}\DecValTok{5}\NormalTok{) \}}
\NormalTok{start\_time }\OtherTok{\textless{}{-}} \FunctionTok{Sys.time}\NormalTok{()}
\FunctionTok{duermete\_un\_minuto}\NormalTok{()}
\NormalTok{end\_time }\OtherTok{\textless{}{-}} \FunctionTok{Sys.time}\NormalTok{()}
\NormalTok{end\_time }\SpecialCharTok{{-}}\NormalTok{ start\_time}
\end{Highlighting}
\end{Shaded}

\begin{verbatim}
## Time difference of 5.249091 secs
\end{verbatim}

Podemos ver que se mide el tiempo de ejecución de la función duérmete un
minuto encerrando la linea que la ejecuta entre el encendido y el
apagado del método sys.time.

\hypertarget{muxe9todo-tictoc}{%
\subsection{\texorpdfstring{\uline{\textbf{MÉTODO
TICTOC}}}{MÉTODO TICTOC}}\label{muxe9todo-tictoc}}

Este es un metodo que debe ser instalado aparte a traves del instalador
de paquetes del programa RStudio ya que no viene de forma predeterminada
en la instalacion de R. Este metodo proviene de las bibliotecas de
Octave/Matlab y ofrece las mismas capacidades que sys.time con algunas
comodidades extra. Un ejemplo de uso de tictoc podria ser:

\begin{Shaded}
\begin{Highlighting}[]
\FunctionTok{library}\NormalTok{(tictoc) }\CommentTok{\#importamos el metodo}
\FunctionTok{tic}\NormalTok{(}\StringTok{"sleeping"}\NormalTok{)}
\NormalTok{A}\OtherTok{\textless{}{-}}\DecValTok{20}
\FunctionTok{print}\NormalTok{(}\StringTok{"dormire una siestita..."}\NormalTok{)}
\end{Highlighting}
\end{Shaded}

\begin{verbatim}
## [1] "dormire una siestita..."
\end{verbatim}

\begin{Shaded}
\begin{Highlighting}[]
\FunctionTok{Sys.sleep}\NormalTok{(}\DecValTok{2}\NormalTok{)}
\FunctionTok{print}\NormalTok{(}\StringTok{"...suena el despertador"}\NormalTok{)}
\end{Highlighting}
\end{Shaded}

\begin{verbatim}
## [1] "...suena el despertador"
\end{verbatim}

\begin{Shaded}
\begin{Highlighting}[]
\FunctionTok{toc}\NormalTok{()}
\end{Highlighting}
\end{Shaded}

\begin{verbatim}
## sleeping: 2.15 sec elapsed
\end{verbatim}

Lo bueno del modulo tictoc es que nos permite medir el rendimiento de
solo un fragmento del codigo y no todo.

\hypertarget{muxe9todo-rbenchmark}{%
\subsection{\texorpdfstring{\uline{\textbf{MÉTODO
RBENCHMARK}}}{MÉTODO RBENCHMARK}}\label{muxe9todo-rbenchmark}}

Este metodo podria decirse que es una version mejorada del metodo
sys.time (se autodescribe como ``un simple contenedor alrededor de
system.time''). Un ejemplo de este podria ser:

\begin{Shaded}
\begin{Highlighting}[]
\FunctionTok{library}\NormalTok{(rbenchmark)}
\CommentTok{\# lm crea una regresi?n lineal}
\FunctionTok{benchmark}\NormalTok{(}\StringTok{"lm"} \OtherTok{=}\NormalTok{ \{}
\NormalTok{X }\OtherTok{\textless{}{-}} \FunctionTok{matrix}\NormalTok{(}\FunctionTok{rnorm}\NormalTok{(}\DecValTok{1000}\NormalTok{), }\DecValTok{100}\NormalTok{, }\DecValTok{10}\NormalTok{)}
\NormalTok{y }\OtherTok{\textless{}{-}}\NormalTok{ X }\SpecialCharTok{\%*\%} \FunctionTok{sample}\NormalTok{(}\DecValTok{1}\SpecialCharTok{:}\DecValTok{10}\NormalTok{, }\DecValTok{10}\NormalTok{) }\SpecialCharTok{+} \FunctionTok{rnorm}\NormalTok{(}\DecValTok{100}\NormalTok{)}
\NormalTok{b }\OtherTok{\textless{}{-}} \FunctionTok{lm}\NormalTok{(y }\SpecialCharTok{\textasciitilde{}}\NormalTok{ X }\SpecialCharTok{+} \DecValTok{0}\NormalTok{)}\SpecialCharTok{$}\NormalTok{coef}
\NormalTok{\},}
\StringTok{"pseudoinverse"} \OtherTok{=}\NormalTok{ \{}
\NormalTok{X }\OtherTok{\textless{}{-}} \FunctionTok{matrix}\NormalTok{(}\FunctionTok{rnorm}\NormalTok{(}\DecValTok{1000}\NormalTok{), }\DecValTok{100}\NormalTok{, }\DecValTok{10}\NormalTok{)}
\NormalTok{y }\OtherTok{\textless{}{-}}\NormalTok{ X }\SpecialCharTok{\%*\%} \FunctionTok{sample}\NormalTok{(}\DecValTok{1}\SpecialCharTok{:}\DecValTok{10}\NormalTok{, }\DecValTok{10}\NormalTok{) }\SpecialCharTok{+} \FunctionTok{rnorm}\NormalTok{(}\DecValTok{100}\NormalTok{)}
\NormalTok{b }\OtherTok{\textless{}{-}} \FunctionTok{solve}\NormalTok{(}\FunctionTok{t}\NormalTok{(X) }\SpecialCharTok{\%*\%}\NormalTok{ X) }\SpecialCharTok{\%*\%} \FunctionTok{t}\NormalTok{(X) }\SpecialCharTok{\%*\%}\NormalTok{ y}
\NormalTok{\},}
\StringTok{"linear system"} \OtherTok{=}\NormalTok{ \{}
\NormalTok{X }\OtherTok{\textless{}{-}} \FunctionTok{matrix}\NormalTok{(}\FunctionTok{rnorm}\NormalTok{(}\DecValTok{1000}\NormalTok{), }\DecValTok{100}\NormalTok{, }\DecValTok{10}\NormalTok{)}
\NormalTok{y }\OtherTok{\textless{}{-}}\NormalTok{ X }\SpecialCharTok{\%*\%} \FunctionTok{sample}\NormalTok{(}\DecValTok{1}\SpecialCharTok{:}\DecValTok{10}\NormalTok{, }\DecValTok{10}\NormalTok{) }\SpecialCharTok{+} \FunctionTok{rnorm}\NormalTok{(}\DecValTok{100}\NormalTok{)}
\NormalTok{b }\OtherTok{\textless{}{-}} \FunctionTok{solve}\NormalTok{(}\FunctionTok{t}\NormalTok{(X) }\SpecialCharTok{\%*\%}\NormalTok{ X, }\FunctionTok{t}\NormalTok{(X) }\SpecialCharTok{\%*\%}\NormalTok{ y)}
\NormalTok{\},}
\AttributeTok{replications =} \DecValTok{1000}\NormalTok{,}
\AttributeTok{columns =} \FunctionTok{c}\NormalTok{(}\StringTok{"test"}\NormalTok{, }\StringTok{"replications"}\NormalTok{, }\StringTok{"elapsed"}\NormalTok{,}
\StringTok{"relative"}\NormalTok{, }\StringTok{"user.self"}\NormalTok{, }\StringTok{"sys.self"}\NormalTok{))}
\end{Highlighting}
\end{Shaded}

\begin{verbatim}
##            test replications elapsed relative user.self sys.self
## 3 linear system         1000    1.42    1.000      0.63     0.05
## 1            lm         1000    9.86    6.944      5.30     0.16
## 2 pseudoinverse         1000    2.02    1.423      0.88     0.05
\end{verbatim}

La ejecucion de este metodo para medir rendimientos nos provee de una
tabla comparativa entre tiempos de uso de diferentes fragmentos de
codigo.

\hypertarget{muxe9todo-microbenchmark}{%
\subsection{\texorpdfstring{\uline{\textbf{MÉTODO
MICROBENCHMARK}}}{MÉTODO MICROBENCHMARK}}\label{muxe9todo-microbenchmark}}

Este metodo ofrece muchas comodidades y funcionalidades adicionales (a
pesar de estar en estado beta). Se destaca en microbenchmark la
posibilidad de poder graficar el rendimiento de diferentes partes de
codigo (o codigos distintos) en diagramas de violin. El siguiente
ejemplo demuestra esto:

\begin{Shaded}
\begin{Highlighting}[]
\FunctionTok{library}\NormalTok{(microbenchmark)}
\FunctionTok{set.seed}\NormalTok{(}\DecValTok{2017}\NormalTok{)}
\NormalTok{n }\OtherTok{\textless{}{-}} \DecValTok{10000}
\NormalTok{p }\OtherTok{\textless{}{-}} \DecValTok{100}
\NormalTok{X }\OtherTok{\textless{}{-}} \FunctionTok{matrix}\NormalTok{(}\FunctionTok{rnorm}\NormalTok{(n}\SpecialCharTok{*}\NormalTok{p), n, p)}
\NormalTok{y }\OtherTok{\textless{}{-}}\NormalTok{ X }\SpecialCharTok{\%*\%} \FunctionTok{rnorm}\NormalTok{(p) }\SpecialCharTok{+} \FunctionTok{rnorm}\NormalTok{(}\DecValTok{100}\NormalTok{)}
\NormalTok{check\_for\_equal\_coefs }\OtherTok{\textless{}{-}} \ControlFlowTok{function}\NormalTok{(values) \{}
\NormalTok{tol }\OtherTok{\textless{}{-}} \FloatTok{1e{-}12}
\NormalTok{max\_error }\OtherTok{\textless{}{-}} \FunctionTok{max}\NormalTok{(}\FunctionTok{c}\NormalTok{(}\FunctionTok{abs}\NormalTok{(values[[}\DecValTok{1}\NormalTok{]] }\SpecialCharTok{{-}}\NormalTok{ values[[}\DecValTok{2}\NormalTok{]]),}
\FunctionTok{abs}\NormalTok{(values[[}\DecValTok{2}\NormalTok{]] }\SpecialCharTok{{-}}\NormalTok{ values[[}\DecValTok{3}\NormalTok{]]),}
\FunctionTok{abs}\NormalTok{(values[[}\DecValTok{1}\NormalTok{]] }\SpecialCharTok{{-}}\NormalTok{ values[[}\DecValTok{3}\NormalTok{]])))}
\NormalTok{max\_error }\SpecialCharTok{\textless{}}\NormalTok{ tol}
\NormalTok{\}}
\NormalTok{mbm }\OtherTok{\textless{}{-}} \FunctionTok{microbenchmark}\NormalTok{(}\StringTok{"lm"} \OtherTok{=}\NormalTok{ \{ b }\OtherTok{\textless{}{-}} \FunctionTok{lm}\NormalTok{(y }\SpecialCharTok{\textasciitilde{}}\NormalTok{ X }\SpecialCharTok{+} \DecValTok{0}\NormalTok{)}\SpecialCharTok{$}\NormalTok{coef \},}
\StringTok{"pseudoinverse"} \OtherTok{=}\NormalTok{ \{}
\NormalTok{b }\OtherTok{\textless{}{-}} \FunctionTok{solve}\NormalTok{(}\FunctionTok{t}\NormalTok{(X) }\SpecialCharTok{\%*\%}\NormalTok{ X) }\SpecialCharTok{\%*\%} \FunctionTok{t}\NormalTok{(X) }\SpecialCharTok{\%*\%}\NormalTok{ y}
\NormalTok{\},}
\StringTok{"linear system"} \OtherTok{=}\NormalTok{ \{}
\NormalTok{b }\OtherTok{\textless{}{-}} \FunctionTok{solve}\NormalTok{(}\FunctionTok{t}\NormalTok{(X) }\SpecialCharTok{\%*\%}\NormalTok{ X, }\FunctionTok{t}\NormalTok{(X) }\SpecialCharTok{\%*\%}\NormalTok{ y)}
\NormalTok{\},}
\AttributeTok{check =}\NormalTok{ check\_for\_equal\_coefs)}
\NormalTok{mbm}
\end{Highlighting}
\end{Shaded}

\begin{verbatim}
## Unit: milliseconds
##           expr      min       lq     mean   median        uq      max neval
##             lm 474.2527 562.1249 896.9026 671.5177  941.2877 3445.384   100
##  pseudoinverse 603.1891 699.5419 935.2912 811.0135 1014.6646 3260.366   100
##  linear system 360.7366 415.9243 573.6044 462.3142  571.7951 2819.641   100
\end{verbatim}

\begin{Shaded}
\begin{Highlighting}[]
\FunctionTok{library}\NormalTok{(ggplot2)}
\FunctionTok{autoplot}\NormalTok{(mbm)}
\end{Highlighting}
\end{Shaded}

\begin{verbatim}
## Coordinate system already present. Adding new coordinate system, which will replace the existing one.
\end{verbatim}

\includegraphics{InformeRFinal_files/figure-latex/unnamed-chunk-4-1.pdf}

\hypertarget{resoluciuxf3n-de-consignas-del-muxf3dulo}{%
\section{\texorpdfstring{\uline{\textbf{RESOLUCIÓN DE CONSIGNAS DEL
MÓDULO}}}{RESOLUCIÓN DE CONSIGNAS DEL MÓDULO}}\label{resoluciuxf3n-de-consignas-del-muxf3dulo}}

\hypertarget{ejercicio-n1---generar-un-vector-secuencia}{%
\subsection{\texorpdfstring{\uline{\textbf{Ejercicio N°1 - Generar un
vector
secuencia}}}{Ejercicio N°1 - Generar un vector secuencia}}\label{ejercicio-n1---generar-un-vector-secuencia}}

En este ejercicio se compararan los rendimientos de dos codigos (uno
usando el metodo FOR y otro usando el metodo nativo de R para
secuencias: seq) con la misma finalidad: generar un vector secuencia de
valores (en este caso desde el 2 hasta el 100.000 con un salto entre
valores de 2). Se utilizara el metodo sys.time para medir los
rendimientos de cada uno por separado.

\hypertarget{secuencias-generadas-con-for}{%
\subsubsection{SECUENCIAS GENERADAS CON
``FOR''}\label{secuencias-generadas-con-for}}

\begin{Shaded}
\begin{Highlighting}[]
\CommentTok{\#SECUENCIAS GENERADAS CON FOR}
\NormalTok{A}\OtherTok{\textless{}{-}}\FunctionTok{c}\NormalTok{()}
\NormalTok{start\_time }\OtherTok{\textless{}{-}} \FunctionTok{Sys.time}\NormalTok{()}

\ControlFlowTok{for}\NormalTok{ (i }\ControlFlowTok{in} \DecValTok{1}\SpecialCharTok{:}\DecValTok{100000}\NormalTok{) \{ A[i] }\OtherTok{\textless{}{-}}\NormalTok{ (i}\SpecialCharTok{*}\DecValTok{2}\NormalTok{)\}}
\FunctionTok{head}\NormalTok{ (A)}\CommentTok{\#MUESTRA LOS PRIMEROS VALORES DE A}
\end{Highlighting}
\end{Shaded}

\begin{verbatim}
## [1]  2  4  6  8 10 12
\end{verbatim}

\begin{Shaded}
\begin{Highlighting}[]
\FunctionTok{tail}\NormalTok{(A)}\CommentTok{\#MUESTRA LOS ULTIMOS VALORES DE A}
\end{Highlighting}
\end{Shaded}

\begin{verbatim}
## [1] 199990 199992 199994 199996 199998 200000
\end{verbatim}

\begin{Shaded}
\begin{Highlighting}[]
\NormalTok{end\_time }\OtherTok{\textless{}{-}} \FunctionTok{Sys.time}\NormalTok{()}
\NormalTok{end\_time }\SpecialCharTok{{-}}\NormalTok{ start\_time}
\end{Highlighting}
\end{Shaded}

\begin{verbatim}
## Time difference of 0.4522152 secs
\end{verbatim}

\hypertarget{secuencias-generadas-con-seq}{%
\subsubsection{SECUENCIAS GENERADAS CON
``SEQ''}\label{secuencias-generadas-con-seq}}

\begin{Shaded}
\begin{Highlighting}[]
\CommentTok{\#SECUENCIAS GENERADAS CON R}

\NormalTok{start\_time }\OtherTok{\textless{}{-}} \FunctionTok{Sys.time}\NormalTok{()}

\NormalTok{A }\OtherTok{\textless{}{-}} \FunctionTok{seq}\NormalTok{(}\DecValTok{2}\NormalTok{,}\DecValTok{100000}\NormalTok{, }\DecValTok{2}\NormalTok{)}
\FunctionTok{head}\NormalTok{ (A)}
\end{Highlighting}
\end{Shaded}

\begin{verbatim}
## [1]  2  4  6  8 10 12
\end{verbatim}

\begin{Shaded}
\begin{Highlighting}[]
\FunctionTok{tail}\NormalTok{(A)}
\end{Highlighting}
\end{Shaded}

\begin{verbatim}
## [1]  99990  99992  99994  99996  99998 100000
\end{verbatim}

\begin{Shaded}
\begin{Highlighting}[]
\NormalTok{end\_time }\OtherTok{\textless{}{-}} \FunctionTok{Sys.time}\NormalTok{()}
\NormalTok{end\_time }\SpecialCharTok{{-}}\NormalTok{ start\_time}
\end{Highlighting}
\end{Shaded}

\begin{verbatim}
## Time difference of 0.08122802 secs
\end{verbatim}

Para ambos metodos, al final de su ejecucion podemos observar los
primeros y ultimos valores del vector secuencia A y el tiempo que se
tardo en ejecutar el codigo. Notamos que el metodo integrado en R
``seq'' es notoriamente mas rapido que ``for'' debido a que es un metodo
optimizado para hacer secuencias unicamente.

\hypertarget{ejercicio-n2---implementaciuxf3n-de-una-serie-fibonacci}{%
\subsection{\texorpdfstring{\uline{\textbf{Ejercicio N°2 -
Implementación de una Serie
Fibonacci}}}{Ejercicio N°2 - Implementación de una Serie Fibonacci}}\label{ejercicio-n2---implementaciuxf3n-de-una-serie-fibonacci}}

Una serie de Fibonacci es tal que comienza con 0 y 1, los siguientes
valores de la serie se calculan como la suma de los dos numeros
anteriores.

\hypertarget{definiciuxf3n-matemuxe1tica-recurrente}{%
\subsubsection{Definición Matemática
Recurrente:}\label{definiciuxf3n-matemuxe1tica-recurrente}}

Esta es la ecuacion caracteristica de la Serie de Fibonacci: \[ 
  f_{0} = 0
\] \[ 
  f_{1} = 1
\] \[ 
  f_{n+1} = f_{n} + f_{n-1}
\] Los primeros terminos de la serie son: 0, 1, 1, 2, 3, 5, 8, 13, 21,
34, 55, 89, 144, 233, 377, 610, 987, 1.597, 2.584, 4.181, 6.765, 10.946,
17.711, 28.657, 46.368, \ldots{}

En este ejercicio se vera cuantas iteraciones se necesitan para generar
un número de la serie mayor que 1.000.000. Ademas mediremos el
rendimiento del codigo a traves del uso de la biblioteca tictoc.

\begin{Shaded}
\begin{Highlighting}[]
\FunctionTok{library}\NormalTok{(tictoc)}

\FunctionTok{tic}\NormalTok{()}

\NormalTok{f1}\OtherTok{\textless{}{-}}\DecValTok{0}
\NormalTok{f2}\OtherTok{\textless{}{-}}\DecValTok{1}
\NormalTok{N}\OtherTok{\textless{}{-}}\DecValTok{0}
\NormalTok{vec}\OtherTok{\textless{}{-}} \FunctionTok{c}\NormalTok{(f1,f2)}
\NormalTok{f3}\OtherTok{\textless{}{-}}\DecValTok{0}

\ControlFlowTok{while}\NormalTok{(f3 }\SpecialCharTok{\textless{}=} \DecValTok{1000000}\NormalTok{) \{}
  
\NormalTok{  N}\OtherTok{\textless{}{-}}\NormalTok{(N}\SpecialCharTok{+}\DecValTok{1}\NormalTok{)}
\NormalTok{  f3}\OtherTok{\textless{}{-}}\NormalTok{ f1 }\SpecialCharTok{+}\NormalTok{ f2}
\NormalTok{  vec}\OtherTok{\textless{}{-}} \FunctionTok{c}\NormalTok{(vec,f3)}
  
\NormalTok{  f1}\OtherTok{\textless{}{-}}\NormalTok{ f2}
\NormalTok{  f2}\OtherTok{\textless{}{-}}\NormalTok{ f3}
  
\NormalTok{\}}

\NormalTok{N}
\end{Highlighting}
\end{Shaded}

\begin{verbatim}
## [1] 30
\end{verbatim}

\begin{Shaded}
\begin{Highlighting}[]
\NormalTok{vec}
\end{Highlighting}
\end{Shaded}

\begin{verbatim}
##  [1]       0       1       1       2       3       5       8      13      21
## [10]      34      55      89     144     233     377     610     987    1597
## [19]    2584    4181    6765   10946   17711   28657   46368   75025  121393
## [28]  196418  317811  514229  832040 1346269
\end{verbatim}

\begin{Shaded}
\begin{Highlighting}[]
\FunctionTok{toc}\NormalTok{()}
\end{Highlighting}
\end{Shaded}

\begin{verbatim}
## 0.2 sec elapsed
\end{verbatim}

A traves del valor del ultimo valor que adquiere N podemos observar que
se necesitan 30 iteraciones minimamente antes de superar el valor
1000000.

\hypertarget{ejercicio-n3---ordenaciuxf3n-de-un-vector-por-muxe9todo-burbuja}{%
\subsection{\texorpdfstring{\uline{\textbf{Ejercicio N°3 - Ordenación de
un vector por método
burbuja}}}{Ejercicio N°3 - Ordenación de un vector por método burbuja}}\label{ejercicio-n3---ordenaciuxf3n-de-un-vector-por-muxe9todo-burbuja}}

En esta consigna compararemos dos codigos a traves del metodo
microbenchmark, ambos con la finalidad de ordenar una lista de menor a
mayor. El primero a analizar sera el metodo de Ordenmiento Directo u
Ordenacion Burbuja (Bubble Sort en inglés), mientras que el segundo sera
un metodo nativo de R que realiza el mismo fin, ``sort''. Ademas y
gracias a microbenchmark, podremos graficar los tiempos de
implementacion de los dos codigos.

Para hacer las pruebas tomaremos una muestra de 50 numeros aleatorios
entre 1 y 20000.

\begin{Shaded}
\begin{Highlighting}[]
\FunctionTok{library}\NormalTok{(microbenchmark)}


\CommentTok{\# Tomo una muestra de 10 números ente 1 y 100}
\NormalTok{x}\OtherTok{\textless{}{-}}\FunctionTok{sample}\NormalTok{(}\DecValTok{1}\SpecialCharTok{:}\DecValTok{20000}\NormalTok{,}\DecValTok{50}\NormalTok{)}
\NormalTok{x}\CommentTok{\# lo ponemos para ver la lista creada}
\end{Highlighting}
\end{Shaded}

\begin{verbatim}
##  [1]  3243  9782  2413 15926 12751  2380  3044 12132   997  7221   481  9699
## [13] 12714 14640 18598 13111 15854  1397  3188  8198  2759 16751  6770 10996
## [25] 18643  4405  6570 16980  8935 15234   696  8498  7979  3723  9201  1022
## [37] 11086  3110  9792  5157  3592  1775  1066 14424 14623  7527 19691  6663
## [49] 17136 12019
\end{verbatim}

\begin{Shaded}
\begin{Highlighting}[]
\CommentTok{\# Creo una funcion para ordenar}

\NormalTok{mbm}\OtherTok{\textless{}{-}} \FunctionTok{microbenchmark}\NormalTok{(}
  \CommentTok{\#METODO ORDENAMIENTO DIRECTO}
  \StringTok{"burbuja"} \OtherTok{=}\NormalTok{ \{}
    
\NormalTok{  burbuja }\OtherTok{\textless{}{-}} \ControlFlowTok{function}\NormalTok{(x)\{}
\NormalTok{  n}\OtherTok{\textless{}{-}}\FunctionTok{length}\NormalTok{(x)}
  \ControlFlowTok{for}\NormalTok{(j }\ControlFlowTok{in} \DecValTok{1}\SpecialCharTok{:}\NormalTok{(n}\DecValTok{{-}1}\NormalTok{))\{}
    \ControlFlowTok{for}\NormalTok{(i }\ControlFlowTok{in} \DecValTok{1}\SpecialCharTok{:}\NormalTok{(n}\SpecialCharTok{{-}}\NormalTok{j))\{}
      \ControlFlowTok{if}\NormalTok{(x[i]}\SpecialCharTok{\textgreater{}}\NormalTok{x[i}\SpecialCharTok{+}\DecValTok{1}\NormalTok{])\{}
\NormalTok{        temp}\OtherTok{\textless{}{-}}\NormalTok{x[i]}
\NormalTok{        x[i]}\OtherTok{\textless{}{-}}\NormalTok{x[i}\SpecialCharTok{+}\DecValTok{1}\NormalTok{]}
\NormalTok{        x[i}\SpecialCharTok{+}\DecValTok{1}\NormalTok{]}\OtherTok{\textless{}{-}}\NormalTok{temp}
\NormalTok{      \}}
\NormalTok{    \}}
\NormalTok{  \}}
  \FunctionTok{return}\NormalTok{(x)}
\NormalTok{\}}
\NormalTok{res}\OtherTok{\textless{}{-}}\FunctionTok{burbuja}\NormalTok{(x)  }
    
\NormalTok{  \}, }\CommentTok{\#Termina el metodo de ordenamiento directo en el vector mbm}


\CommentTok{\#METODO R}
\StringTok{"sort"}\OtherTok{=}\NormalTok{ \{}
  
  \FunctionTok{sort}\NormalTok{(x)}
  
\NormalTok{\} }\CommentTok{\#Termina el metodo R}
\NormalTok{)}

\FunctionTok{sort}\NormalTok{(x) }\CommentTok{\#lo ponemos solo para ver la lista ordenada}
\end{Highlighting}
\end{Shaded}

\begin{verbatim}
##  [1]   481   696   997  1022  1066  1397  1775  2380  2413  2759  3044  3110
## [13]  3188  3243  3592  3723  4405  5157  6570  6663  6770  7221  7527  7979
## [25]  8198  8498  8935  9201  9699  9782  9792 10996 11086 12019 12132 12714
## [37] 12751 13111 14424 14623 14640 15234 15854 15926 16751 16980 17136 18598
## [49] 18643 19691
\end{verbatim}

\begin{Shaded}
\begin{Highlighting}[]
\NormalTok{mbm}
\end{Highlighting}
\end{Shaded}

\begin{verbatim}
## Unit: microseconds
##     expr   min     lq     mean median     uq     max neval
##  burbuja 632.4 659.70 2432.997 714.25 1007.4 96480.2   100
##     sort 181.4 187.95  331.983 204.40  415.4  2997.5   100
\end{verbatim}

\begin{Shaded}
\begin{Highlighting}[]
\FunctionTok{library}\NormalTok{(ggplot2)}\CommentTok{\#importamos la biblioteca de graficos}
\FunctionTok{autoplot}\NormalTok{(mbm)}\CommentTok{\#graficamos los rendimientos}
\end{Highlighting}
\end{Shaded}

\begin{verbatim}
## Coordinate system already present. Adding new coordinate system, which will replace the existing one.
\end{verbatim}

\includegraphics{InformeRFinal_files/figure-latex/unnamed-chunk-8-1.pdf}

Por el grafico podemos observar que el metodo burbuja es menos eficaz
que el metodo nativo de R (esto se debe otra vez debido a que este
ultimo ha sido optimizado para esta tarea mientras que el otro es un
codigo creado a mano por nosotros para realizar la tarea). El grafico de
sort() esta mas a la izquierda indicando un menor tiempo de comienzo de
realizacion de la tarea y tambien termina antes que el de Ordemiento
Directo, el cual no solo comienza despues sino que se prolonga mas a lo
largo del tiempo debido a las multiples comparaciones que debe ejecutar
para ordenar los numeros.

\hypertarget{ejercicio-n4---progresiuxf3n-geomuxe9trica-del-covid-19}{%
\subsection{\texorpdfstring{\uline{\textbf{Ejercicio N°4 - Progresión
geométrica del
COVid-19}}}{Ejercicio N°4 - Progresión geométrica del COVid-19}}\label{ejercicio-n4---progresiuxf3n-geomuxe9trica-del-covid-19}}

En este ejercicio utilizaremos un modelo matematico para determinar como
se expande un virus en una pandemia. Usaremos los datos publicos subidos
a internet de casos en Argentina. Determinaremos en que fecha se
contagiaran 40 millones de personas.

Utilizaremos un factor de contagio F=1.62 y una poblacion contagiada
activa inicial de 51778 personas (extraido de la pagina oficial del
ministerio de salud).

\begin{Shaded}
\begin{Highlighting}[]
\NormalTok{f1}\OtherTok{\textless{}{-}} \DecValTok{51778}
\NormalTok{f2}\OtherTok{\textless{}{-}} \DecValTok{0}
\NormalTok{N}\OtherTok{\textless{}{-}} \DecValTok{0}
\NormalTok{vec}\OtherTok{\textless{}{-}} \FunctionTok{c}\NormalTok{(f1)}
\NormalTok{F }\OtherTok{\textless{}{-}} \FloatTok{1.62}

\ControlFlowTok{while}\NormalTok{(f2 }\SpecialCharTok{\textless{}=} \DecValTok{40000000}\NormalTok{) \{}
  
\NormalTok{  N}\OtherTok{\textless{}{-}}\NormalTok{(N}\SpecialCharTok{+}\DecValTok{1}\NormalTok{)}
\NormalTok{  f2}\OtherTok{\textless{}{-}}\NormalTok{ F}\SpecialCharTok{*}\NormalTok{f1}
\NormalTok{  vec}\OtherTok{\textless{}{-}} \FunctionTok{c}\NormalTok{(vec,f2)}
  
\NormalTok{  f1}\OtherTok{\textless{}{-}}\NormalTok{ f2}
  
\NormalTok{\}}

\NormalTok{N}
\end{Highlighting}
\end{Shaded}

\begin{verbatim}
## [1] 14
\end{verbatim}

\begin{Shaded}
\begin{Highlighting}[]
\NormalTok{vec}
\end{Highlighting}
\end{Shaded}

\begin{verbatim}
##  [1]    51778.00    83880.36   135886.18   220135.62   356619.70   577723.91
##  [7]   935912.74  1516178.64  2456209.39  3979059.21  6446075.93 10442643.00
## [13] 16917081.66 27405672.29 44397189.11
\end{verbatim}

Por el valor de N, el virus tardara 14 dias en infectar a 40 millones de
personas.

\hypertarget{importar-datos-de-archivo-.csv-y-graficar}{%
\subsubsection{\texorpdfstring{\uline{\textbf{Importar datos de archivo
.csv y
graficar}}}{Importar datos de archivo .csv y graficar}}\label{importar-datos-de-archivo-.csv-y-graficar}}

Con esta seccion aprederemos a importar datos de un archivo .csv y luego
operar con ellos. Importaremos el archivo oficial de registro de
contagios de covid-19 en la pandemia en Argentina, ``casos.csv'' (el
cual esta subido a la pagina de la catedra). Lo descargaremos y
ubicaremos en la carpeta ``Descargas'' de nuestra computadora. Para
importarlo deberemos seguir los siguientes pasos:

\begin{enumerate}
\def\labelenumi{\arabic{enumi}.}
\item
  Seleccionar ``File''--\textgreater{} ``Import
  Dataset''--\textgreater{}``From Text (readr)''.
\item
  Con el boton ``Browse'' ubicaremos el archivo en nuestra computadora y
  lo cargaremos.
\item
  En la seccion ``Delimiter'' tendremos que seleccionar ``Semicolon''.
\item
  Daremos click en ``Import'' para importar.
\item
  Para invocarlo en nuestro codigo deberemos ejecutar el siguiente
  codigo:

\begin{Shaded}
\begin{Highlighting}[]
\FunctionTok{library}\NormalTok{(readr)}
\NormalTok{casos }\OtherTok{\textless{}{-}} \FunctionTok{read\_delim}\NormalTok{(}\StringTok{"C:/Users/Nico/Downloads/casos.csv"}\NormalTok{,}
\AttributeTok{delim =} \StringTok{";"}\NormalTok{, }\AttributeTok{escape\_double =} \ConstantTok{FALSE}\NormalTok{, }\AttributeTok{col\_types =} \FunctionTok{cols}\NormalTok{(}\StringTok{\textasciigrave{}}\AttributeTok{Covid Argentina}\StringTok{\textasciigrave{}} \OtherTok{=} \FunctionTok{col\_date}\NormalTok{(}\AttributeTok{format =} \StringTok{"\%m/\%d/\%Y"}\NormalTok{)),}
\AttributeTok{trim\_ws =} \ConstantTok{TRUE}\NormalTok{)}
\end{Highlighting}
\end{Shaded}

\begin{verbatim}
## New names:
## * `` -> `...2`
## * `` -> `...3`
\end{verbatim}

\begin{verbatim}
## Warning: One or more parsing issues, see `problems()` for details
\end{verbatim}
\end{enumerate}

Para poder graficar los contagiados al pasar de las semanas debremos
ejecutar un comando tipo ``plot'', nativo de R:

\begin{Shaded}
\begin{Highlighting}[]
\FunctionTok{plot}\NormalTok{(casos}\SpecialCharTok{$}\NormalTok{...}\DecValTok{2}\NormalTok{, }\AttributeTok{main=}\StringTok{"Contagio 2020"}\NormalTok{, }\AttributeTok{ylab=}\StringTok{"Casos positivos"}\NormalTok{, }\AttributeTok{xlab=}\StringTok{"Semanas"}\NormalTok{)}
\end{Highlighting}
\end{Shaded}

\begin{verbatim}
## Warning in xy.coords(x, y, xlabel, ylabel, log): NAs introducidos por coerción
\end{verbatim}

\includegraphics{InformeRFinal_files/figure-latex/unnamed-chunk-11-1.pdf}

\bibliography{mybibfile.bib}


\end{document}
